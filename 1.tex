\chapter{}\label{chp:1}
The problem contains several linear inequalities, and thus can best be solved as an SMT, by Z3 for example. We have a number of trucks, let's say $n$. For each truck $i$, with $1 \leq i \leq n$ we in turn introduce a variable for every pallet type $x \in X$, where $X$ is the set of pallet types. We have $x_i :=$ the amount of pallets delivered by truck $i$ of type $x$. For example, in the current exercise we get the following variables from $X$:
\begin{itemize}
    \item $c_i$: number of cheese pallets delivered by truck $i$
    \item $b_i$: number of beer pallets delivered by truck $i$
    \item $w_i$: number of wine pallets delivered by truck $i$
    \item $d_i$: number of soft drink pallets delivered by truck $i$
    \item $p_i$: number of potato chips pallets delivered by truck $i$
\end{itemize}

Each pallet type has its own weight, and thus for each type $x$ we introduce a constant $x_w$, which denotes the weight of a pallet of type $x$. We also have a maximum load weight $T_c$ for all trucks. In this exercise we have $T_c = 8500$. This translates to the constraint specified in Equation~\ref{maxload}
\begin{equation}
    \label{maxload}
    \bigwedge^{n}_{i=1}\bigg(\big(\sum_{x \in X} x_i \cdot x_w\big) \leq T_c \bigg)
\end{equation}

We also have that every truck can carry at most $T_p$, where $T_p$ is a set amount. Note that in the case of the current exercise $T_p = 8$. This constraint is specified in Equation~\ref{mp}.
\begin{equation}
    \label{mp}
    \bigwedge^{n}_{i=1}\bigg(\sum_{x \in X} x_i \leq T_p\bigg)
\end{equation}

Next, there are a set amount of pallets to be delivered for each pallet type, except beer. To generalize this, we introduce for each pallet type $x$ a constant $x_d$ which denotes the number of pallets of type $x$ that need to be delivered. For all pallets which have no specified quantity for delivery we let $x_d := -1$. Let $X^{\oplus}$ denote the set of pallet types $x$ for which $x_d \geq 0$. We specify this constraint formally in the Equation~\ref{delivery}.
\begin{equation}
    \begin{aligned}
    \label{delivery}
    \bigwedge_{x \in X^{\oplus}}\bigg(\sum^{n}_{i=1} x_i = x_d)\bigg)
    \end{aligned}
\end{equation}

Let $m$ trucks be allowed to deliver alcohol, for some $0 \leq m \leq n$. Note that which trucks are allowed to deliver alcohol is not important as they all have the same load capacity. Then $n - m$ trucks are not allowed to deliver alcohol, and thus they should deliver zero pallets containing alcohol. Let $x_a \in \{\textit{True}, \textit{False}\}$ denote whether pallet type $x$ contains alcohol, and let $X^{a}$ denote the set of pallet types $x$ for which $x_a = \textit{True}$. We get the constraint specified in Equation~\ref{alcohol}.
\begin{equation}
    \label{alcohol}
    \bigwedge^n_{i=m+1} \bigg(\bigwedge_{x \in X^a} \big( x_i = 0 \big)\bigg)
\end{equation}

Under these constraints we want to maximize the number of pallets delivered for a given set of pallet types $P$. In the exercise we have $P = \{b\}$, as we have to maximize the amount of beer pallets delivered. The general maximization functions then become:
\begin{equation}
    \begin{aligned}
    \textbf{Maximize: } \sum^n_{i=1} x_i && \text{for each }x \in P
    \end{aligned}
\end{equation}

The extra requirement introduced in (b) states that each truck can contain at most two pallets of soft drinks. We can generalize this by introducing a variable $x_{\max}$ for every pallet type $x$, such that $x_{\max}$ denotes the maximum amount of pallets of that type allowed per truck. This means that for all trucks $1 \leq i \leq n$ we have $x_i \leq x_{\max}$. In total, we get the constraint in Equation~\ref{extra}. Note that for the exercise in (b) we have $d_{\max} = 2$.
\begin{equation}
    \label{extra}
    \bigwedge^n_{i=1}\bigg( \bigwedge_{x \in X} x_i \leq x_{\max}\bigg)
\end{equation}




\begin{table}[!ht]
  \begin{tabular}{r | r r r r r}
    B = 18\\
    Truck &  C &  B &  W &  D &  P\\\hline
        1 &  0 &  6 &  0 &  0 &  0\\
        2 &  0 &  4 &  3 &  0 &  1\\
        3 &  0 &  3 &  3 &  1 &  1\\
        4 &  0 &  5 &  2 &  0 &  0\\
        5 &  1 &  0 &  0 &  5 &  2\\
        6 &  3 &  0 &  0 &  4 &  1\\
  \end{tabular}
  \caption{A pallet distribution that satisfies the constraints and maximizes the amount of beer pallets delivered in (a).}
  \label{tab:1_table}
\end{table}

\chapter{}\label{chp:4}
Some ambiguity exists in the given algorithm, since no indentation or closing of the \textit{for} loop is present, it is important to emphasize that the algorithm has been interpreted as shown in \Cref{alg:4_improved}. The condition in the \textit{if-then} statement leading to the program crash has been extracted to an extra variable $c$, the use of this variable will become apparent in the analysis of the results.
\begin{algorithm}
  \caption{The way the given algorithm has been interpreted}
  \label{alg:4_improved}
  \begin{algorithmic}[1]
    \State $a := 1$;
    \State $b := 1$;
    \For{$i := 1 \to 10$}
      \If{$t$}
        \State $a := a + 2b$;
        \State $b := b + i$;
      \Else
        \State $b := a + b$;
        \State $a := a + i$;
      \EndIf
      \State $c := (b = 600 + n)$
      \If{$c$}
        \State crash
      \EndIf
    \EndFor
  \end{algorithmic}
\end{algorithm}

The algorithm may be converted to SMT format by introducing variables $a_{i}$, $b_{i}$, $c_{i}$ and $t_{i}$ which represent the value of the value $a$, the value $b$, the crash condition and the unknown test from the original algorithm respectively, after the $i$\textsuperscript{th} iteration of the loop. Many of the constraints for this problem depend on the number of iterations in the \textit{for} loop, which is represented by $imax = 10$, both in the equations presented as in the script generating the requirements file as outlined in \Cref{app:4_gen.py}. In the script the minimal and maximal values for $n$ are also factored out to $nmin = 1$ and $nmax = 10$, just as the excluded values for $n$ are factored out to $nexclude = [1, 2, 3, 4, 8, 9, 10]$. This implementation makes the script and therefore the generated requirements file highly general.

The initial values for $a$ and $b$ can be expressed by \Cref{eqn:4_initial}, while the statements in the \textit{if-then-else} can be expressed by \Cref{eqn:4_loop}. The $t_{i} \implies \ldots$ here implements the updates on variables $a$ and $b$ in iteration $i$ of the loop if the test $t$ yields true in that iteration. Conversely the $\neg t_{i} \implies \ldots$ implements the updates on variables $a$ and $b$ in iteration $i$ of the loop if the test $t$ yields false in that iteration. In the SMT-LIB v2 syntax this can be condensed to an \textit{ite} statement as shown in \Cref{app:4_req.smt}.
\begin{equation}
  \label{eqn:4_initial}
  (a_{0} = 1) \wedge (b_{0} = 1)
\end{equation}

\begin{equation}
  \label{eqn:4_loop}
  \begin{aligned}
    \bigwedge_{i = 1}^{imax}
      \Big(t_{i} \implies (a_{i} = a_{i - 1} + 2b_{i - 1}) \wedge (b_{i} = b_{i - 1} + i)\Big) \wedge \\
      \Big(\neg t_{i} \implies (b_{i} = a_{i - 1} + b_{i - 1}) \wedge (a_{i} = a_{i - 1} + i)\Big)
    \end{aligned}
\end{equation}

The conditions under which the program crashes can be captured in the variables $c_{i}$ as demonstrated in \Cref{eqn:4_condcrash}, while the condition that the program actually crashes is a large disjunction over all variables $c_{i}$, as shown in \Cref{eqn:4_docrash}. In these requirements the variables $c_{i}$ indicate if the program crashes during iteration $i$.
\begin{equation}
  \label{eqn:4_condcrash}
  \bigwedge_{i = 1}^{imax}
    c_{i} \iff (b_{i} = 600 + n)
\end{equation}

\begin{equation}
  \label{eqn:4_docrash}
  \bigvee_{i = 1}^{imax} c_{i}
\end{equation}

Finally the values of $n$ that need to be checked need to be bounded, this happens by means of \Cref{eqn:4_valn}, where the set in the $\bigwedge$ lists the excluded values of $n$. Initially this set is empty. Each time an unsafe value for $n$ is discovered it is added to this set, to determine if all other values for $n$ are safe. All non-excluded values for $n$ are safe iff the requirements are unsatisfiable, since then no sequence of outcomes of the unknown test can be found that make the condition that the program crashes in any iteration true. If the requirements are satisfiable, a sequence of outcomes for the unknown test can be found leading to a crash and an example path will be generated.
\begin{equation}
  \label{eqn:4_valn}
  (n \geq nmin) \wedge (n \leq nmax) \wedge
  \bigwedge_{i \in \{nexclude\}} n \neq i
\end{equation}

A Python script to generate the SMT-LIB v2 file containing the conjunction of all requirements is written and the contents are presented in \Cref{app:4_gen.py}. The requirements file generated by this script for the specific problem can be found in \Cref{app:4_req.smt}. Running the Z3 solver on this file produces an output which can be interpreted by another python script that generates a table of the backtrace of the program, in case a possibility to crash the program is found. This script can be found in \Cref{app:4_table.py}.

Since the possible values of $n$ start with the set $\{i \in \mathbb{Z} \mid 1 \leq i \leq 10\}$ and an unsatisfiable result is only found when the set of exclusions for $n$ consists of $\{1, 2, 3, 4, 8, 9, 10\}$, it can be concluded that the safe values for $n$ are $\{i \in \mathbb{Z} \mid 1 \leq i \leq 10\} \setminus \{1, 2, 3, 4, 8, 9, 10\} = \{5, 6, 7\}$. The values for $n$ in $\{1, 2, 3, 4, 8, 9, 10\}$ are therefore unsafe and an example of a path leading to a crash is shown in \Cref{tab:4_crashings} for each of these values. In the $t$ column the output of the unknown test is shown needed to lead to a crash, while in the $c$ column it can be seen in which iteration the program crashes.

\newcommand{\IVcrash}{\rowcolor{lightgray}}
\begin{center}
  \bottomcaption{The values for $n$ for which the algorithm is unsafe, and an example path leading to a crash for each of them}
  \label{tab:4_crashings}
  \begin{supertabular}{r l r r l}
    \hline\hline
    n = 1 \\
     i &     t &    a &    b &     c \\\hline
     0 &       &    1 &    1 &       \\
     1 &  true &    3 &    2 & false \\
     2 & false &    5 &    5 & false \\
     3 &  true &   15 &    8 & false \\
     4 & false &   19 &   23 & false \\
     5 &  true &   65 &   28 & false \\
     6 &  true &  121 &   34 & false \\
     7 & false &  128 &  155 & false \\
     8 &  true &  438 &  163 & false \\
     9 & false &  447 &  601 &  true $\Rightarrow$ crash\\\IVcrash
    10 & false &  457 & 1048 & false \\
    \hline\hline
    n = 2 \\
     i &     t &    a &    b &     c \\\hline
     0 &       &    1 &    1 &       \\
     1 & false &    2 &    2 & false \\
     2 & false &    4 &    4 & false \\
     3 &  true &   12 &    7 & false \\
     4 &  true &   26 &   11 & false \\
     5 & false &   31 &   37 & false \\
     6 & false &   37 &   68 & false \\
     7 & false &   44 &  105 & false \\
     8 &  true &  254 &  113 & false \\
     9 &  true &  480 &  122 & false \\
    10 & false &  490 &  602 &  true $\Rightarrow$ crash\\
    \hline\hline
    n = 3 \\
     i &     t &    a &    b &     c \\\hline
     0 &       &    1 &    1 &       \\
     1 &  true &    3 &    2 & false \\
     2 &  true &    7 &    4 & false \\
     3 &  true &   15 &    7 & false \\
     4 & false &   19 &   22 & false \\
     5 &  true &   63 &   27 & false \\
     6 & false &   69 &   90 & false \\
     7 &  true &  249 &   97 & false \\
     8 & false &  257 &  346 & false \\
     9 & false &  266 &  603 &  true $\Rightarrow$ crash\\\IVcrash
    10 & false &  276 &  869 & false \\
    \hline\hline
    n = 4 \\
     i &     t &    a &    b &     c \\\hline
     0 &       &    1 &    1 &       \\
     1 & false &    2 &    2 & false \\
     2 &  true &    6 &    4 & false \\
     3 &  true &   14 &    7 & false \\
     4 &  true &   28 &   11 & false \\
     5 & false &   33 &   39 & false \\
     6 &  true &  111 &   45 & false \\
     7 & false &  118 &  156 & false \\
     8 &  true &  430 &  164 & false \\
     9 & false &  439 &  594 & false \\
    10 &  true & 1627 &  604 &  true $\Rightarrow$ crash\\
    \hline\hline
    n = 8 \\
     i &     t &    a &    b &     c \\\hline
     0 &       &    1 &    1 &       \\
     1 & false &    2 &    2 & false \\
     2 &  true &    6 &    4 & false \\
     3 &  true &   14 &    7 & false \\
     4 &  true &   28 &   11 & false \\
     5 &  true &   50 &   16 & false \\
     6 &  true &   82 &   22 & false \\
     7 &  true &  126 &   29 & false \\
     8 & false &  134 &  155 & false \\
     9 &  true &  444 &  164 & false \\
    10 & false &  454 &  608 &  true $\Rightarrow$ crash\\
    \hline\hline
    n = 9 \\
     i &     t &    a &    b &     c \\\hline
     0 &       &    1 &    1 &       \\
     1 &  true &    3 &    2 & false \\
     2 &  true &    7 &    4 & false \\
     3 &  true &   15 &    7 & false \\
     4 &  true &   29 &   11 & false \\
     5 & false &   34 &   40 & false \\
     6 &  true &  114 &   46 & false \\
     7 & false &  121 &  160 & false \\
     8 &  true &  441 &  168 & false \\
     9 & false &  450 &  609 &  true $\Rightarrow$ crash\\\IVcrash
    10 & false &  460 & 1059 & false \\
    \hline\hline
    n = 10 \\
     i &     t &    a &    b &     c \\\hline
     0 &       &    1 &    1 &       \\
     1 & false &    2 &    2 & false \\
     2 & false &    4 &    4 & false \\
     3 &  true &   12 &    7 & false \\
     4 & false &   16 &   19 & false \\
     5 & false &   21 &   35 & false \\
     6 & false &   27 &   56 & false \\
     7 &  true &  139 &   63 & false \\
     8 &  true &  265 &   71 & false \\
     9 & false &  274 &  336 & false \\
    10 & false &  284 &  610 &  true $\Rightarrow$ crash\\
  \end{supertabular}
\end{center}
